\documentclass[11pt]{article}
\usepackage[utf8]{inputenc}
\usepackage[T1]{fontenc}
\usepackage{fixltx2e}
\usepackage{graphicx}
\usepackage{grffile}
\usepackage{longtable}
\usepackage{wrapfig}
\usepackage{rotating}
\usepackage[normalem]{ulem}
\usepackage{amsmath}
\usepackage{textcomp}
\usepackage{amssymb}
\usepackage{capt-of}
\usepackage{hyperref}
\author{Chris Howe-Jones}
\date{15th December 2015}
\title{The Never Changing Face of Immutability}
\hypersetup{
 pdfauthor={Chris Howe-Jones},
 pdftitle={The Never Changing Face of Immutability},
 pdfkeywords={},
 pdfsubject={},
 pdfcreator={Emacs 25.1.1 (Org mode 8.3.6)},
 pdflang={English}}
\begin{document}

\maketitle


\section*{Warning!!}
\label{sec:orgheadline1}

\begin{itemize}
\item There will be a Lisp!
\item There will be Entomology!
\item There will be History!
\end{itemize}
\begin{NOTES}
\begin{itemize}
\item 1st law of Clojure talks
\item Any talk with Clojure in it must have some entomology
\end{itemize}
\end{NOTES}


\section*{The Never Changing Face of Immutability}
\label{sec:orgheadline29}

\includegraphics[width=.9\linewidth]{./immutable-defined.png}

\subsubsection*{Who am I?}
\label{sec:orgheadline2}

Name:      \texttt{Chris Howe-Jones}

Job Title: \texttt{Technical Navigator}

Twitter:   \texttt{@agile\_geek}

Github:    \texttt{github.com/chrishowejones}

Blog:      \texttt{chrishowejones.wordpress.com}

\subsubsection*{Credentials}
\label{sec:orgheadline3}

\begin{itemize}
\item 28 years of pushing data around
\item Procedural/OOP/FP
\item Architecture \& Design
\item RAD/Agile/Lean
\item CTO
\end{itemize}

\subsection*{History Lesson}
\label{sec:orgheadline4}

\includegraphics[width=.9\linewidth]{./John-McCarthy.jpg}

\begin{NOTES}
\begin{itemize}
\item Who is this?
\item John McCarthy
\begin{itemize}
\item developed Lisp
\item influenced design of ALGOL
\item invented GC
\item created term AI
\item first to suggest publicly the idea of utility computing
\item credited with developing an early form of time-sharing
\end{itemize}
\end{itemize}
\end{NOTES}

\subsection*{Once upon a time..}
\label{sec:orgheadline8}

\includegraphics[width=.9\linewidth]{./book-keepers.jpg}

Book Keeping
\begin{itemize}
\item List of entries in a ledger
\end{itemize}
\begin{itemize}
\item No 'crossing out'!
\end{itemize}

\subsubsection*{Dawn of Computing}
\label{sec:orgheadline5}

\includegraphics[width=.9\linewidth]{./EDSAC.jpg}

\begin{itemize}
\item Math
\item Transient storage
\end{itemize}

\begin{NOTES}
\begin{itemize}
\item EDSAC - Electronic Delay Storage Automatic Calculator
\item Cambridge 1949 - early general purpose electronic programmable computer (ENIAC 1946 was 1st)
\item Storage - mecury delay lines, derated vacuum tubes for logic
\item n 1950, M. V. Wilkes and Wheeler used EDSAC to solve a differential equation relating to gene frequencies in a paper by Ronald Fisher. This represents the first use of a computer for a problem in the field of biology.
\item In 1951, Miller and Wheeler used the machine to discover a 79-digit prime – the largest known at the time.
\item In 1952, Sandy Douglas developed OXO, a version of noughts and crosses (tic-tac-toe) for the EDSAC, with graphical output to a VCR97 6" cathode ray tube. This may well have been the world's first video game.
\end{itemize}
\end{NOTES}

\subsubsection*{60's-90's}
\label{sec:orgheadline6}

\includegraphics[width=.9\linewidth]{./1960s-computer.jpg}

\begin{itemize}
\item Spot the expense?
\end{itemize}
\begin{itemize}
\item Memory
\end{itemize}
\begin{itemize}
\item Tape
\end{itemize}
\begin{itemize}
\item Disk
\end{itemize}


\subsubsection*{21st Century}
\label{sec:orgheadline7}

\includegraphics[width=.9\linewidth]{./pair-programming.png}

Spot the expense?
\begin{itemize}
\item Developers
\end{itemize}
Cheap resources: SSD/Disk, Memory, CPU


\subsection*{And..}
\label{sec:orgheadline11}

\includegraphics[width=.9\linewidth]{./fry-so.jpg}

\subsubsection*{In place computing}
\label{sec:orgheadline9}

\includegraphics[width=.9\linewidth]{./core_memory.jpg}

\begin{itemize}
\item Update data in place
\end{itemize}
\begin{itemize}
\item Reuse expensive real estate
\end{itemize}

\begin{NOTES}
\begin{itemize}
\item Magnetic core memory 1955-75
\item Core uses tiny magnetic toroids (rings), the cores, through which wires are threaded to write and read information.
\item Each core represents one bit of information.
\item Magnetized in 2 directions (clockwise/counterclockwise) to represent 1 or 0
\end{itemize}
\end{NOTES}

\subsubsection*{RDBMS}
\label{sec:orgheadline10}

\includegraphics[width=.9\linewidth]{./disk-pack.jpg}

\begin{itemize}
\item Data updated
\end{itemize}
\begin{itemize}
\item Values overwritten
\end{itemize}
\begin{itemize}
\item Reuse memory and disk
\end{itemize}

\begin{NOTES}
\begin{itemize}
\item Disk pack - invented 1965
\item IBM Engineers - Thomas G. Leary and R. E. Pattison
\item Probably about 50MB on this one.
\end{itemize}
\end{NOTES}

\subsection*{Result?}
\label{sec:orgheadline16}

In place oriented programming (PLOP) relies on\ldots{}

\subsubsection*{Mutation}
\label{sec:orgheadline12}

\includegraphics[width=.9\linewidth]{./mutation.jpg}

\subsubsection*{Which leads to..}
\label{sec:orgheadline13}

\includegraphics[width=.9\linewidth]{./complect.png}

\subsubsection*{Complect}
\label{sec:orgheadline14}

\includegraphics[width=.9\linewidth]{./plaiting.jpg}

\begin{itemize}
\item Complecting Identity \& Value
\end{itemize}
\begin{itemize}
\item Especially RDBMS, OOP
\end{itemize}
\begin{itemize}
\item Pessimistic concurrency strategies
\end{itemize}

\subsubsection*{What's changed?}
\label{sec:orgheadline15}
\url{./historical_cost_graph5.gif}

\begin{itemize}
\item Computing capacity has increased by a million fold!
\end{itemize}

\subsection*{Immutability (and values) to the rescue!}
\label{sec:orgheadline19}

\includegraphics[width=.9\linewidth]{./lambda-man.jpg}

\subsubsection*{Values}
\label{sec:orgheadline17}

\includegraphics[width=.9\linewidth]{./values.jpeg}
\begin{itemize}
\item Values are generic
\item Values are easy to fabricate
\item Drives reuse
\item Values aggregate to values
\item Distributable
\end{itemize}

\subsubsection*{Isn't copying values inefficient?}
\label{sec:orgheadline18}

\includegraphics[width=.9\linewidth]{./clojure-persistent-data-structures-sharing.png}

\begin{itemize}
\item Structural sharing
\item For example in Clojure:
\begin{itemize}
\item persistent bit-partitioned vector trie
\item 32 node tries
\item Wide shallow trees
\end{itemize}
\end{itemize}

\subsection*{What does it look like?}
\label{sec:orgheadline22}

\begin{itemize}
\item Immutable by default
\item Explicit state change
\item Database as a value
\end{itemize}

\begin{NOTES}
\begin{itemize}
\item Make state change obvious
\item Pass a snapshot of the database as a value
\begin{itemize}
\item always remote
\end{itemize}
\item Lack of Basis from database
\begin{itemize}
\item consistency across long term conversations
\item what does update mean?
\end{itemize}
\end{itemize}
\end{NOTES}

\subsubsection*{ClojureScript on the client}
\label{sec:orgheadline20}

\begin{verbatim}
(def initial-state
  {:event {:event/name "" :event/speaker ""} :server-state nil})
\end{verbatim}
\begin{verbatim}
(defn- event-form
  [ui-channel {:keys [event/name event/speaker] :as event}]
  [:table.table
   [:tr
    [:td [:label "Event name:"]]
    [:td [:input {:type :text
                  :placeholder "Event name..."
                  :defaultValue event/name
                  :on-change (send-value! ui-channel m/->ChangeEventName)}]]]
   [:tr
    [:td [:label "Speaker:"]]
    [:td [:input {:type :text
                  :placeholder "Speaker..."
                  :defaultValue event/speaker
                  :on-change (send-value! ui-channel m/->ChangeEventSpeaker)}]]]
   [:tr
    [:td
     [:button.btn.btn-success
      {:on-click (send! ui-channel (m/->CreateEvent))}
      "Go"]]]])
\end{verbatim}

\begin{verbatim}
(defrecord ChangeEventName [name])

(defrecord ChangeEventSpeaker [speaker])

(defrecord CreateEvent [event])

(defrecord CreateEventResults [body])
\end{verbatim}
\begin{verbatim}
(extend-protocol Message
  m/ChangeEventName
  (process-message [{:keys [name]} app]
    (assoc-in app [:event :event/name] name)))
;; redacted for clarity ...

(extend-protocol EventSource
  m/CreateEvent
  (watch-channels [_ {:keys [event]
                      :as app}]
    #{(rest/create-event event)}))

(extend-protocol Message
  m/CreateEventResults
  (process-message [response app]
    (assoc app :server-state (-> response :body))))
\end{verbatim}

\subsubsection*{Efficiency}
\label{sec:orgheadline21}

\includegraphics[width=.9\linewidth]{./todomvc-perf-comparison.png}

\subsection*{Clojure on the server}
\label{sec:orgheadline23}

\begin{verbatim}
(defn- handle-query
  [db-conn]
  (fn [{req-body :body-params}]
    {:body (case (:type req-body)
             :get-events (data/get-events db-conn)
             :create-event (data/create-entity db-conn (:txn-data req-body)))}))


(defn app [dbconn]
  (-> (routes
       (GET "/" [] home-page)
       (POST "/q" []
             (handle-query dbconn))
       (resources "/"))
      (wrap-restful-format :formats [:edn :transit-json])
      (rmd/wrap-defaults (-> rmd/site-defaults
                             (assoc-in [:security :anti-forgery] false)))))
\end{verbatim}

\subsection*{Datomic for Data}
\label{sec:orgheadline27}

\includegraphics[width=.9\linewidth]{./datomic-architecture.png}

\begin{itemize}
\item App get's its own query, comms, memory- Each App is a peer
\end{itemize}

\begin{NOTES}
\begin{itemize}
\item Apps are peers
\item Transactor broadcasts txns to peers
\item Peers cache data locally
\end{itemize}
\end{NOTES}

\subsubsection*{Database as a value}
\label{sec:orgheadline24}

\begin{center}
\begin{tabular}{llll}
Entity & Attribute & Value & Time\\
\hline
Fiona & likes & Ruby & 01/06/2015\\
Dave & likes & Haskell & 25/09/2015\\
Fiona & likes & Clojure & 15/12/2015\\
 &  &  & \\
\hline
 &  &  & \\
\end{tabular}
\end{center}

\begin{itemize}
\item Effectively DB is local
\item Datalog query language
\end{itemize}
\begin{verbatim}
[:find ?e :where [?e :likes “Clojure”]]
\end{verbatim}

\begin{NOTES}
\begin{itemize}
\item Ask connection for database - it returns a value representing the db
\item This is because datoms are immutable - new versions thru time
\item Can invoke your own code from query engine as data is just normal data structures (lists, maps, etc.)
\item Assertions and retractions of facts (Datoms)
\end{itemize}
\end{NOTES}

\subsubsection*{Schema}
\label{sec:orgheadline25}

\begin{verbatim}
 ;;event
 {
  :db/id                 #db/id[:db.part/db]
  :db/ident              :event/name
  :db/cardinality        :db.cardinality/one
  :db/valueType          :db.type/string
  :db/unique             :db.unique/identity
  :db.install/_attribute :db.part/db
  }
 {
  :db/id                 #db/id[:db.part/db]
  :db/ident              :event/description
  :db/cardinality        :db.cardinality/one
  :db/valueType          :db.type/string
  :db.install/_attribute :db.part/db
  }
 {
  :db/id                 #db/id[:db.part/db]
  :db/ident              :event/location
  :db/cardinality        :db.cardinality/one
  :db/valueType          :db.type/ref
  :db.install/_attribute :db.part/db
  }
...
\end{verbatim}
\begin{verbatim}
;;location
 {
  :db/id                 #db/id[:db.part/db]
  :db/ident              :location/postCode
  :db/cardinality        :db.cardinality/one
  :db/valueType          :db.type/string
  :db.install/_attribute :db.part/db
  }
 {
  :db/id                 #db/id[:db.part/db]
  :db/ident              :location/description
  :db/cardinality        :db.cardinality/one
  :db/valueType          :db.type/string
  :db.install/_attribute :db.part/db
  }
...
\end{verbatim}
\subsubsection*{Persistence}
\label{sec:orgheadline26}

\begin{verbatim}
(defn create-entity
  "Takes transaction data and returns the resolved tempid"
  [conn tx-data]
  (let [had-id (contains? tx-data ":db/id")
        data-with-id (if had-id
                       tx-data
                       (assoc tx-data :db/id #db/id[:db.part/user -1000001]))
        tx @(d/transact conn [data-with-id])]
    (if had-id (tx-data ":db/id")
        (d/resolve-tempid (d/db conn) (:tempids tx)
                          (d/tempid :db.part/user -1000001)))))
\end{verbatim}
\begin{verbatim}
(defn get-events [db]
  (d/pull-many db [:*]
               (->> (d/q '{:find [?event-id]
                           :where [[?event-id :event/name]]}
                         db)
                    (map first))))
\end{verbatim}

\subsection*{Conclusion?}
\label{sec:orgheadline28}
\includegraphics[width=.9\linewidth]{./you-cant-step.jpg}
\begin{itemize}
\item Immutability simplifies
\item State as function call stack
\item Mostly pure functions
\begin{itemize}
\item Easier to test \& reason about
\end{itemize}
\item Time as first class concept
\item Easier to distribute
\end{itemize}

\section*{Resources}
\label{sec:orgheadline30}

\begin{itemize}
\item Rich Hickey talks -
\begin{itemize}
\item 'The Value of Values'
\item 'The Language of the System'
\item 'Simple Made Easy'
\item 'Clojure, Made Simple'
\item 'The Database as a Value'
\item 'The Language of Systems'
\end{itemize}
\item Moseley and Marks - Out of the Tar Pit
\item Kris Jenkins
\begin{itemize}
\item 'ClojureScript - Architecting for Scale' (Clojure eXchange 2015)
\end{itemize}
\end{itemize}

\begin{NOTES}
\begin{itemize}
\item History
\begin{itemize}
\item book keeping - double entry. Didn't change in place.
\item 50's, 60's memory expensive resource (dates? picture of large old machine)
\item Swapping instructions in and out of memory - tape -> disk
\item 70's, 80's and 90's secondary storage expensive - rise of RDBMS
\item memory still reasonably expensive
\item In place computing as resources scarce
\item 00's and 2010's disk cheaper, memory very cheap.
\item in parallel the rise of OOP - objects with data and behaviour
\end{itemize}
\item Why immutability?
\begin{itemize}
\item What does mutation bring (picture of three eyed fish from Simpsons \_ other pop culture references)
\item Can't stand in same river twice (where is origin of quote?)
\item Complecting the concepts of identity and value particularly OO and RDBMS in trad. use.
\item Issues of concurrency. Complex values are changed underneath you.
\item Optimisations - (dig out graph of Om compared with React.js)
\end{itemize}
\item What does it look like?
\begin{itemize}
\item Examples in:
\begin{itemize}
\item Clojurescript - UI state as a value
\item Clojure - server state as value and a chain of functions
\item Datomic - database as a value - local cache, peer to peer
\end{itemize}
\end{itemize}
\end{itemize}
\end{NOTES}
\end{document}
